\chapter{Micro-architectural Counters}

The current RISC-V gcc toolchain provides access to 16 named ``uarch" counters.\footnote{The future plan of record is to move these counters into the memory-mapped access region. This will open up the ability to track a much larger number of counters, as desired.}

\begin{table}[htp]
\caption{Uarch Counters}
\begin{center}
\begin{tabular}{|c|c|}
\hline
Number & Event \\
\hline
\hline
0 & Committed Branch Mispredictions \\
\hline
1 & Committed Branches \\
\hline
\end{tabular}
\end{center}
\label{table:uarchcounters}
\end{table}%

The counters can be modified in {\tt dpath.scala} to track events of interest.

{\bf Note:} the counters can be quite large (64-bits each), and it is recommended that alternative methods be used if a silicon is the end-product. A design that can multiplex counters is recommended. 

\section{Reading UArch Counters in Software}

The Code Example \ref{ref:code_uarch} demonstrates how to read the value of any CSR register from software.
  
  
\begin{center}
\begin{minipage}{0.66\textwidth}
\begin{lstlisting}[caption=Reading a CSR register]
#define read_csr_safe(reg) ({ register long __tmp asm("a0"); \   
  asm volatile ("csrr %0, " #reg : "=r"(__tmp)); \               
  __tmp; })             
  
  long csr_cycle   = read_csr_safe(cycle);
  long csr_instr   = read_csr_safe(instret);
  long csr_uarch0  = read_csr_safe(uarch0);
  ...
  long csr_uarch15 = read_csr_safe(uarch15);
  
\end{lstlisting}\label{ref:code_uarch}
\end{minipage}
\end{center}


